% Noah Scarbrough, Samuel Dunn
% CS 483, Fall 2019
% Group project proprosal.

\documentclass{article}

\title{CS 483 Project Proposal}
\author{Noah Scarbrough, Samuel Dunn}
\date{10-17-2019}

\begin{document}
\maketitle

\section{Data Set}
Our data set for this project will be the cards used in trading card game,
Magic: The Gathering. The card text will be treated as a document. 
With nearly 20,000 unique cards in existence, there is a wealth of data available.
If this proves to bee too large to process within the scope of this assignment we will
trim our scraping to be the subset of cards that are legal in the modern format. 
Along with the document, each tuple will additionally contain other attributes for the card, such as its mana cost and rarity. 
This data will be scraped from the official database site for the game.

\section{Feature: Advanced Search}
Our first feature will be an advanced search engine. 
The user will have the option of using the standard search or the advanced search. 
The standard search will simply return the cards most relevant to the keyword query across all components of the card. 
Advanced search will be more intricate and designed to assist with deck building. The advanced search will allow the user to hone in on matching particular aspects of a card. 
Elements such as color identity, converted mana cost, type, supported formats and so forth.

\section{Feature: Related Cards}
Another feature we plan to include is a display of relevant cards. 
When a card is chosen from the results, it will open a page that 
contains not only information about the card, but also give a list 
of links for related cards. Related cards will be determined by mining
information from popular deckbuilding sites, such as tappedout.net, for
cards commonly used in conjunction to the card the user has selected. 
If unable to glean precise information from mining, we may fall back
to suggesting cards with similar features, mana cost, set legality, and so forth.

\end{document}
